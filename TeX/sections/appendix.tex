\section{Appendix}

\subsection{Memorylessness (for distributions)}

A random variable \( X \) is memoryless if it satisfies
\begin{equation}
    P(X > s + t \mid X > s) = P(X > t)
\end{equation}

\subsection{Markov property (/ Memorylessness property)}

A stochastic process \( {\{ X_n \}}_{n \geq 0} \) is said to have the Markov property if
\begin{equation}
    P(X_{n+1} \leq x \mid X_n = x_n, X_{n-1} = x_{n-1}, \ldots, X_0 = x_0) = P(X_{n+1} \leq x \mid X_n = x_n)
\end{equation}
for all \( n \geq 0 \) and \( x, x_n, x_{n-1}, \ldots, x_0 \in \mathbb{R} \).
The future state of the process depends only on the present state and not on the past states.

\subsection{Semi-Markov property}

A stochastic process \( {\{ X_n \}}_{n \geq 0} \) is said to have the semi-Markov property if
\begin{equation}
    \begin{aligned}
        P(X_{n+1} \leq x \mid X_n = x_n, T_n = t_n, X_{n-1} = x_{n-1}, T_{n-1} = t_{n-1}, \ldots, X_0 = x_0, T_0 = t_0)
        \\ =
        P(X_{n+1} \leq x \mid X_n = x_n, T_n = t_n)
    \end{aligned}
\end{equation}
for all \( n \geq 0 \) and \( x, x_n, x_{n-1}, \ldots, x_0 \in \mathbb{R} \).
The future state of the process depends only on the present state and the time spent in that state, not on the past states or times.
The semi-Markov property is a generalization of the Markov property, where the time spent in each state is also taken into account.

\subsection{Stochastic process}

A stochastic process is a collection of random variables \( \{ X(t) \} \) indexed by time \( t \).
The process can be discrete-time or continuous-time, depending on whether the index set is discrete or continuous.
The transition probabilities can be time-homogeneous (the same at all times) or time-inhomogeneous (varying with time).

\subsection{Markov process / Markov chain}

A Markov process is a stochastic process that satisfies the Markov property.

\subsection{Semi-Markov process}

A semi-Markov process is a stochastic process that satisfies the semi-Markov property.
It is characterized by a sequence of random variables \( \{ X_n \} \) representing the states of the process and a sequence of random variables \( \{ T_n \} \) representing the sojourn (temporary stay) times in each state.
