\section*{About | Option-critic~\cite{bacon2017option, Sutton1999}}

\( \Omega \to \) Set of all Markov options

\( \omega \to \) A Markov option.\@ \( \omega = (\mathcal{I}_\omega, \pi_\omega, \beta_\omega) \in \Omega \)

\( \mathcal{I}_\omega \to \) Initiation set.\@ \( \mathcal{I}_\omega \subseteq \mathcal{S} \)

\( \pi_\omega \to \) Intra-option policy.\@ \( \pi_\omega: \mathcal{S} \times \mathcal{A} \to [0, 1] \)

\( \beta_\omega \to \) Termination function.\@ \( \beta_\omega: \mathcal{S} \to [0, 1] \)

\( V_\Omega(s) \to \) Value function over options.\@ \( V_\Omega: \mathcal{S} \to \mathbb{R} \)

\( Q_\Omega(s, \omega) \to \) Option-value function.\@ \( Q_\Omega: \mathcal{S} \times \Omega \to \mathbb{R} \)

\( Q_U(s, \omega, a) \to \) Value of executing an action in the context of a state-option pair.\@ \( Q_U: \mathcal{S} \times \Omega \times \mathcal{A} \to \mathbb{R} \)

\( U(\omega, s') \to \) Option-value function upon arrival.\@ \( U: \Omega \times \mathcal{S} \to \mathbb{R} \)

\textbf{Parameterisation:}

\( \pi_{\omega, \theta}(a \mid s) \to \) Intra-option policy of option \( \omega \) parameterised by \( \theta \).
One such parameterisation is
\begin{equation}
    \pi_{\theta, \omega}(a \mid s)
    =
    \frac{\exp(\theta_\omega^T \phi_\omega(s, a))}{\sum_{a'} \exp(\theta_\omega^T \phi_\omega(s, a'))}
    , \quad
    \phi_\omega(s, a)
    \to
    \text{Feature vector for } s, a
    , \quad
    \theta_\omega
    \to
    \text{Parameter vector for } \omega
\end{equation}
\( \beta_{\omega, \vartheta}(s) \to \) Termination function of option \( \omega \) parameterised by \( \vartheta \).
\begin{equation}
    V_\Omega(s)
    =
    \sum_{\omega} \pi_\Omega(\omega \mid s) \, Q_\Omega(s, \omega)
    =
    \mathbb{E}_{\omega \sim \pi_\Omega(\cdot \mid s)} \Big[ Q_\Omega(s, \omega) \Big]
    \label{eq:V_Omega}
\end{equation}

\textbf{Objective:} Goal is to learn options that maximise the expected return in the current task.
\begin{align}
    Q_\Omega(s, \omega)
     & =
    \sum_{a} \pi_{\theta, \omega}(a \mid s) \ Q_U(s, \omega, a)
    \label{eq:Q_Omega}
    \\
    Q_U(s, \omega, a)
     & =
    r(s, a) + \gamma \sum_{s'} P(s' \mid s, a) \ U(\omega, s')
    \label{eq:Q_U}
    \\
    U(\omega, s')
     & =
    \big( 1 - \beta_{\omega, \vartheta}(s') \big) \, Q_\Omega(s', \omega) + \beta_{\omega, \vartheta}(s') \, V_\Omega(s')
    \label{eq:U}
\end{align}

\textbf{Markov chain} with state-option pairs as states: It's one-step transition probability is given by
\begin{equation}
    P(s_{t+1}, \omega_{t+1} \mid s_t, \omega_t)
    =
    \sum_{a} \pi_{\omega_t, \theta}(a \mid s_t) \ P(s_{t+1} \mid s_t, a) \, \Big( \big( 1 - \beta_{\omega_t, \vartheta}(s_{t+1}) \big) \mathbf{1}_{\omega_t = \omega_{t+1}} + \beta_{\omega_t, \vartheta}(s_{t+1}) \, \pi_\Omega(\omega_{t+1} \mid s_{t+1}) \Big)
\end{equation}
This process is homogeneous and ergodic.

\textbf{Intra-Option Policy Gradient Theorem:} Initial condition \( (s_0, \omega_0) \)
\begin{align}
    \frac{\partial Q_\Omega(s, \omega)}{\partial \theta}
     & =
    \sum_{s, \omega} \mu_\Omega(s, \omega \mid s_0, \omega_0) \sum_{a} \frac{\partial \pi_{\omega, \theta}(a \mid s)}{\partial \theta} \ Q_U(s, \omega, a)
    \\
    \text{where }
    \mu_\Omega(s, \omega \mid s_0, \omega_0)
     & =
    \sum_{t=0}^\infty \gamma^t P(s_t = s, \omega_t = \omega \mid s_0, \omega_0)
\end{align}

\textbf{Termination Gradient Theorem:}
Initial condition \( (s_1, \omega_0) \)
\begin{align}
    \frac{\partial Q_\Omega(s, \omega)}{\partial \vartheta}
     & =
    - \sum_{s', \omega} \mu_\Omega(s', \omega \mid s_1, \omega_0) \, \frac{\partial \beta_{\omega, \vartheta}(s')}{\partial \vartheta} \ A_\Omega(s', \omega)
    \\
    \text{where }
    \mu_\Omega(s, \omega \mid s_1, \omega_0)
     & =
    \sum_{t=0}^\infty \gamma^t P(s_t = s, \omega_t = \omega \mid s_1, \omega_0)
    \\
    A_\Omega(s', \omega)
     & =
    Q_\Omega(s', \omega) - V_\Omega(s')
\end{align}
